%\documentclass[a4paper,12pt]{report}
\documentclass[
  11pt,				
  openright,	
  twoside,	
  a4paper,
  brazil,	
  oldfontcommands,
  ]{abntex2}

\usepackage{lmodern}
\usepackage[utf8]{inputenc}
%\usepackage[english]{babel}
\usepackage[T1]{fontenc}
\usepackage[brazil]{babel}

%\usepackage{natbib}
%\usepackage{bibentry}
\usepackage{etoolbox}
\patchcmd{\thebibliography}{\section*{\refname}}{}{}{}
\patchcmd{\thebibliography}{\addcontentsline{toc}{section}{\refname}}{}{}{}

\usepackage{comment}

\usepackage[table,hyperref,x11names]{xcolor}
%\usepackage{colortbl}

\usepackage{pifont}% http://ctan.org/pkg/pifont
\newcommand{\cmark}{\ding{51}}%
\newcommand{\xmark}{\ding{55}}%
\newcommand{\whitespace}{{\color{white}x}}%

\usepackage{graphicx}
\graphicspath{{contrib-fcul-src/figures/}}
\DeclareGraphicsExtensions{.pdf,.jpeg,.png,.eps}
\usepackage{boxedminipage}

\usepackage{pgfplots}
\usepackage{savesym}
\usepackage{amsmath}
\savesymbol{iint}
\usepackage{txfonts}
\restoresymbol{TXF}{iint}

\usepackage{url}
\usepackage{xspace}
\usepackage{array}
\usepackage{amsmath}
\usepackage{tabularx}
\usepackage{makeidx}
\usepackage{fancyhdr}
\usepackage{color}
\usepackage{setspace}
\usepackage{longtable}
\usepackage{multirow}
\usepackage{daytime}
\usepackage{amsmath}
\usepackage{amssymb}
\usepackage[ddmmyyyy]{datetime}
\usepackage{listings}
\usepackage{algorithmicx,float}
\usepackage[lined,ruled,linesnumbered,inoutnumbered,commentsnumbered]{algorithm2e}
\usepackage{algpseudocode}
%\usepackage[colorlinks=true,urlcolor=SteelBlue4,linkcolor=Firebrick4]{hyperref}
%\usepackage[show]{chato-notes}
%\usepackage[hide]{chato-notes}

\usepackage{enumitem}
\newcommand{\subscript}[2]{$#1 _ #2$}

% page borders config
\oddsidemargin -0.6in
\evensidemargin -0.6in
\textwidth 7.5in
\topmargin -0.6in
\textheight 9.5in
\setlength{\headsep}{80pt}

%\setlength{\parindent}{0in}
\setlength{\parskip}{1ex plus 1ex minus 0.2ex}
\renewcommand{\dateseparator}{-}
\renewcommand{\arraystretch}{1.5}
\newdateformat{ukvardate}{\monthname[\THEMONTH]\xspace\THEDAY, \THEYEAR}

\let\oldthebibliography\thebibliography
\let\endoldthebibliography\endthebibliography
\renewenvironment{thebibliography}[1]{
  \begin{oldthebibliography}{#1}
    \setlength{\itemsep}{0em}
    \setlength{\parskip}{0em}
}
{
  \end{oldthebibliography}
}

%\usepackage{ltablex}

\begin{document}

\pagestyle{fancy}

\fancyhf{}
\fancyhead[CO,CE]{\includegraphics[width=4.5cm]{images/logo_projeto_ensino.png}}
\fancyfoot[RE,RO]{\thepage\xspace de \pageref{DocLastPage}}
\fancyfoot[LO,LE]{\footnotesize Formul\'{a}rio abril 2019 (conforme modelo referência disponibilizado no SIPPEE).}

\begin{center}
Formulário de Projeto de Ensino
\end{center}

%%%%%%%%%%%%%%%%%%%%%%%%%%%%%%%%%%%%%%%%%%%%%%%%%%%%%%%%%%%%%%%%%%%%%
% DADOS DO PROJETO
\begin{table}[htp]
\begin{center}
\begin{tabularx}{\textwidth}{|p{5.5cm}|X|}
\hline
\cellcolor{lightgray} \textbf{T\'{i}tulo do Projeto} & Meu Projeto de Ensino \\\hline

\cellcolor{lightgray} \textbf{Datas (dia/mês/ano)} & \hfill \textbf{Início}: 30/04/2019 \hfill \textbf{Fim:} 31/12/2020 \hfill {\color{white}x} \\\hline

\cellcolor{lightgray} \textbf{Proponente} &\hfill \textbf{Nome:} Fulano de Tal \hfill \textbf{Email:} Fulano.de.Tal@gmail.com \hfill {\color{white}x}\\\hline 

\cellcolor{lightgray} \textbf{Campus} &  Porto Alegre-SC \\\hline

\cellcolor{lightgray} \textbf{Curso(s) de vinculação} & Curso 1 \\\hline

\cellcolor{lightgray} \textbf{Área do Conhecimento CNPq} &  Ciências Exatas e da Terra \\\hline

\cellcolor{lightgray} \textbf{Carga Horária Semanal} &  20 horas \\\hline


\cellcolor{lightgray} \textbf{Palavras-chave} {\tiny (máx. 4, separadas por ponto e vírgula)} & Avaliação Híbrida, Formas de Avaliação, Universidade do Século XXI , Diversidade \\\hline

\end{tabularx}
\end{center}
\end{table}
\vspace{-10mm}
%%%%%%%%%%%%%%%%%%%%%%%%%%%%%%%%%%%%%%%%%%%%%%%%%%%%%%%%%%%%%%%%%%%%%

{\footnotesize
O projeto deve ter, no máximo, 10 páginas. Fonte: Times New Roman, 12, espaço entre linhas 1,5
}

%%%%%%%%%%%%%%%%%%%%%%%%%%%%%%%%%%%%%%%%%%%%%%%%%%%%%%%%%%%%%%%%%%%%%
% RESUMO
\begin{table}[H]
\begin{center}
\rowcolors{1}{lightgray}{lightgray}\begin{tabularx}{\textwidth}{|X|}
\hline
{\bf Resumo} {\tiny (somente texto, até 35000 caracteres)}\\\hline
\end{tabularx}
\end{center}
\end{table}
\vspace{-10mm}
%%%%%%%%%%%%%%%%%%%%%%%%%%%%%%%%%%%%%%%%%%%%%%%%%%%%%%%%%%%%%%%%%%%%%

Atualmente, bla bla bla bla bla


%\vspace{5mm}

\noindent {\large \textbf{Equipe Executora}} (Adicionar quantas linhas for necessário)

\begin{table}[H]
\begin{center}
\begin{tabularx}{\textwidth}{|X|p{2.5cm}|p{3.5cm}|p{3.5cm}|p{1.5cm}|}
\cellcolor{lightgray} {\bf Nome} & \cellcolor{lightgray} \textbf{Vínculo} {\tiny \bf (Discente, Docente, TAE ou outro)} & \cellcolor{lightgray} \textbf{Campus} & \cellcolor{lightgray} \textbf{Função} {\tiny \bf (coordenador, colaborador, bolsista, etc.)}  & \cellcolor{lightgray} \textbf{Carga Horária Semanal} \\\hline
Fulano de Tal		  		           & Docente & Porto Alegre-SC  & Coordenador     & 2 horas \\\hline
Alice Maravilha    		   & Discente & Porto Alegre-SC  & Colaboradora 	  & 2 horas \\\hline
Bolsista B(*)			 	          & Discente & Porto Alegre-SC  & Bolsista & 12 horas \\\hline
Voluntário V(*)		 	                & Discente & Porto Alegre-SC  & Voluntário & 12 horas \\\hline
\end{tabularx}
\end{center}
\end{table}
\vspace{-10mm}
\noindent (*) São previstos pelo menos um bolsista (para 2020) e um voluntário, ambos com dedicação de no mínimo 12 horas para o projeto. Bolsistas e voluntários serão selecionados conforme a demanda do projeto.

%%%%%%%%%%%%%%%%%%%%%%%%%%%%%%%%%%%%%%%%%%%%%%%%%%%%%%%%%%%%%%%%%%%%%
% INTRODUCAO
\begin{table}[H]
\begin{center}
\rowcolors{1}{lightgray}{lightgray}\begin{tabularx}{\textwidth}{|X|}
\hline
{\bf 1 Introdução e justificativa} \textit{(Contextualização e importância da temática do projeto, considerando o Projeto Pedagógico do Curso de Graduação do proponente e o Plano de Desenvolvimento Institucional (PDI) da Unipampa, principalmente em atenção aos contextos sociocultural, educacional, econômico e político da região de inserção da Unipampa ou do Campus}\\\hline
\end{tabularx}
\end{center}
\end{table}
\vspace{-10mm}
%%%%%%%%%%%%%%%%%%%%%%%%%%%%%%%%%%%%%%%%%%%%%%%%%%%%%%%%%%%%%%%%%%%%%

A Universidade bla bla bla bla



%%%%%%%%%%%%%%%%%%%%%%%%%%%%%%%%%%%%%%%%%%%%%%%%%%%%%%%%%%%%%%%%%%%%%
% OBJETIVOS
\begin{table}[H]
\begin{center}
\rowcolors{1}{lightgray}{lightgray}\begin{tabularx}{\textwidth}{|X|}
\hline
{\bf 2 Objetivos}\\\hline
{\bf 2.1 Geral}\\\hline
\end{tabularx}
\end{center}
\end{table}
\vspace{-10mm}
%%%%%%%%%%%%%%%%%%%%%%%%%%%%%%%%%%%%%%%%%%%%%%%%%%%%%%%%%%%%%%%%%%%%%

Investigar bla bla bla bla 

%%%%%%%%%%%%%%%%%%%%%%%%%%%%%%%%%%%%%%%%%%%%%%%%%%%%%%%%%%%%%%%%%%%%%
% OBJETIVOS ESPECIFICOS
\begin{table}[H]
\begin{center}
\rowcolors{1}{lightgray}{lightgray}\begin{tabularx}{\textwidth}{|X|}
\hline
{\bf 2.3 Específicos} \textit{(Ação pretendida considerando: a) a temática do projeto e o aprofundamento conceitual (conteúdos);
 c) alternativas à evasão e retenção; d) a interação e integração acadêmica de forma interdisciplinar e contextualizada; e) a interação ensino, pesquisa e extensão; f) a comunicação e ou socialização dos resultados obtido)}\\\hline
\end{tabularx}
\end{center}
\end{table}
\vspace{-10mm}
%%%%%%%%%%%%%%%%%%%%%%%%%%%%%%%%%%%%%%%%%%%%%%%%%%%%%%%%%%%%%%%%%%%%%

\begin{enumerate}[label=(\subscript{e}{{\arabic*}})]

\item avaliar bla bla bla

\item analisar bla bla bla 

\item propor bla bla bla

\end{enumerate}


%%%%%%%%%%%%%%%%%%%%%%%%%%%%%%%%%%%%%%%%%%%%%%%%%%%%%%%%%%%%%%%%%%%%%
% REFERENCIAL TEORICO
\begin{table}[H]
\begin{center}
\rowcolors{1}{lightgray}{lightgray}\begin{tabularx}{\textwidth}{|X|}
\hline
{\bf 3 Referencial teórico}\\\hline
\end{tabularx}
\end{center}
\end{table}
\vspace{-10mm}
%%%%%%%%%%%%%%%%%%%%%%%%%%%%%%%%%%%%%%%%%%%%%%%%%%%%%%%%%%%%%%%%%%%%%

Segundo especialistas, bla bla bla bla bla


%%%%%%%%%%%%%%%%%%%%%%%%%%%%%%%%%%%%%%%%%%%%%%%%%%%%%%%%%%%%%%%%%%%%%
% PRESSUPOSTOS METODOLOGICOS
\begin{table}[H]
\begin{center}
\rowcolors{1}{lightgray}{lightgray}\begin{tabularx}{\textwidth}{|X|}
\hline
{\bf 4 Pressupostos metodológicos} \textit{(metodologias e estratégias que contemplam: a) os procedimentos em relação aos objetivos propostos e temática do projeto; b) a interação e integração acadêmica de forma interdisciplinar e contextualizada; c) a articulação ensino, pesquisa e extensão; d) a Comunicação e ou socialização dos resultados obtidos);} \\\hline
\end{tabularx}
\end{center}
\end{table}
\vspace{-10mm}
%%%%%%%%%%%%%%%%%%%%%%%%%%%%%%%%%%%%%%%%%%%%%%%%%%%%%%%%%%%%%%%%%%%%%

\textbf{Aspectos formativos}.
A equipe executora bla bla bla

\textbf{Sistematização de estudos e elaboração de relatórios}.
Bla bla bla bla 

\textbf{Articulação com a pesquisa}.
Bla bla bla bla

\textbf{Articulação com a extensão}.
Bla bla bla bla

\textbf{Avaliação das atividades exercidas pelos discentes}.
Bla bla bla bla

%%%%%%%%%%%%%%%%%%%%%%%%%%%%%%%%%%%%%%%%%%%%%%%%%%%%%%%%%%%%%%%%%%%%%
% RESULTADOS ESPERADOS
\begin{table}[H]
\begin{center}
\rowcolors{1}{lightgray}{lightgray}\begin{tabularx}{\textwidth}{|X|}
\hline
{\bf Resultados esperados} \textit{(considerar os objetivos geral e específicos)}  \\\hline
\end{tabularx}
\end{center}
\end{table}
\vspace{-10mm}
%%%%%%%%%%%%%%%%%%%%%%%%%%%%%%%%%%%%%%%%%%%%%%%%%%%%%%%%%%%%%%%%%%%%%

\begin{enumerate}[label=(\subscript{r}{{\arabic*}})]

\item Aumentar bla bla bla bla
\item Promover bla bla bla bla 
\item Desenvolver bla bla bla bla
\item Estimular bla bla bla bla
\end{enumerate}

\vfill

{\color{white}x}

%%%%%%%%%%%%%%%%%%%%%%%%%%%%%%%%%%%%%%%%%%%%%%%%%%%%%%%%%%%%%%%%%%%%%
% PARCERIAS
%\begin{comment}
\begin{table}[H]
\begin{center}
\begin{tabularx}{\textwidth}{|X|p{5cm}|}
\multicolumn{2}{l}{\noindent \textbf{Avaliação do projeto}}\\
\cellcolor{lightgray} {\bf Critério} & \cellcolor{lightgray} \textbf{Indicador} \\\hline

\multirow{7}{*}{\shortstack[l]{Processos relativos a adoção das metodologias, estabelecimento de\\ metas e acompanhamento do êxito destas metodologias.(*)}} 
& Média de alunos presentes em aula \\\cline{2-2}
& Percepção dos alunos em relação aos professores/disciplinas \\\cline{2-2}
& Percepção dos docentes em relação a turma \\\cline{2-2}
& Satisfação dos alunos em relação ao método ensino aprendizagem \\\hline

Os estudantes julgam a utilização de diferentes abordagens de avaliação relevantes ao processo de ensino-aprendizagem & Porcentagem de alunos satisfeitos \\\hline

As avaliações realizadas estavam dentro das expectativas dos estudantes
 & Porcentagem de alunos satisfeitos \\\hline
 
\end{tabularx}
\end{center}
\end{table}
\vspace{-10mm}
%\end{comment}
%%%%%%%%%%%%%%%%%%%%%%%%%%%%%%%%%%%%%%%%%%%%%%%%%%%%%%%%%%%%%%%%%%%%%
\noindent (*) Critério global, válido para todos os indicadores.


\vfill

{\color{white}x}


%%%%%%%%%%%%%%%%%%%%%%%%%%%%%%%%%%%%%%%%%%%%%%%%%%%%%%%%%%%%%%%%%%%%%
% REFERENCIAS
\begin{table}[H]
\begin{center}
\rowcolors{1}{lightgray}{lightgray}\begin{tabularx}{\textwidth}{|X|}
\hline
{\bf Referências}\\\hline
\end{tabularx}
\end{center}
\end{table}
\vspace{-10mm}
%%%%%%%%%%%%%%%%%%%%%%%%%%%%%%%%%%%%%%%%%%%%%%%%%%%%%%%%%%%%%%%%%%%%%

\begingroup
\renewcommand{\section}[2]{}%
\renewcommand{\chapter}[2]{}% for other classes

{\normalsize
\bibliographystyle{sbc}
\bibliography{references}
}

\endgroup



\vspace{5mm}
\noindent {\large \textbf{Cronograma de Atividades}}  (Adicionar quantas linhas for necessário)


%%%%%%%%%%%%%%%%%%%%%%%%%%%%%%%%%%%%%%%%%%%%%%%%%%%%%%%%%%%%%%%%%%%%%
% CRONOGRAMA
% ANO 1
\begin{table}[H]
\begin{center}
\begin{tabularx}{\textwidth}{|p{10cm}|X|X|X|X|X|X|X|X|X|X|X|X|X|}
\multicolumn{13}{|c|}{\cellcolor{lightgray} \textbf{Ano 1}} \\\hline
\cellcolor{lightgray} & J & F & M & A & M & J & J & A & S & O & N & D \\
\cellcolor{lightgray}  \textbf{Descri\c{c}\~{a}o da Atividade} & A & E & A & B & A & U & U & G & E & U & O & E \\
\cellcolor{lightgray}  & N & V & R & R & I & N & L & O & T & T & V & Z \\\hline
Revisão da literatura       		&  x  & x & x  & x &  &  &   &   &   &   &   &   \\\hline
Seleção das disciplinas  e apresentação do projeto   	&    &   & x & x &   &   &   &  x &  x  &   &    &   \\\hline
Planejamento dos métodos a 
serem aplicados e avaliados 	&    &   &   &  & x  &   &   & x  &  x  &   &    &   \\\hline
Preparação de material e aplicação dos métodos 		&    &   &   &  & x  & x  & x  &   &  x  &  x & x   &   \\\hline
Coleta e análise de dados para 
avaliação do projeto 			&    &   &    &   &   &   & x  &   &    &   & x   & x  \\\hline
Proposição de novas técnicas 
sistemáticas de avaliação        &    &   &    &   &   &   & x  & x  &    &   &  x  & x  \\\hline
Divulgar e publicar os 
resultados do primeiro ano  	&    &   &   &   &   &   &  &  &  & x & x & x \\\hline
\end{tabularx}
\end{center}
\end{table}
\vspace{-10mm}
%%%%%%%%%%%%%%%%%%%%%%%%%%%%%%%%%%%%%%%%%%%%%%%%%%%%%%%%%%%%%%%%%%%%%

%%%%%%%%%%%%%%%%%%%%%%%%%%%%%%%%%%%%%%%%%%%%%%%%%%%%%%%%%%%%%%%%%%%%%
% CRONOGRAMA
% ANO 2
\begin{table}[H]
\begin{center}
\begin{tabularx}{\textwidth}{|p{10cm}|X|X|X|X|X|X|X|X|X|X|X|X|X|}
\multicolumn{13}{|c|}{\cellcolor{lightgray} \textbf{Ano 2}} \\\hline
\cellcolor{lightgray} & J & F & M & A & M & J & J & A & S & O & N & D \\
\cellcolor{lightgray}  \textbf{Descri\c{c}\~{a}o da Atividade} & A & E & A & B & A & U & U & G & E & U & O & E \\
\cellcolor{lightgray}  & N & V & R & R & I & N & L & O & T & T & V & Z \\\hline
Seleção das disciplinas  e apresentação do projeto   	&    &   & x & x &   &   &   &  x &  x  &   &    &   \\\hline
Planejamento dos métodos a 
serem aplicados e avaliados 	&    &   &  x & x &   &   &   & x  &  x  &   &    &   \\\hline
Preparação de material e aplicação dos métodos 		&    & x  & x  &  & x  & x  & x  &   &  x  &  x & x   &   \\\hline
Coleta e análise de dados para 
avaliação do projeto 			&    &   &    &   &   & x  & x  &   &    &   & x   & x  \\\hline
Proposição de novas técnicas 
sistemáticas de avaliação        &    &   &    &   &   &   & x  & x  &    &   &  x  & x  \\\hline
Divulgar e publicar os 
resultados do primeiro ano  	&    &   &   &   &   &   &  &  &  & x & x & x \\\hline
\end{tabularx}
\end{center}
\end{table}
\vspace{-10mm}
%%%%%%%%%%%%%%%%%%%%%%%%%%%%%%%%%%%%%%%%%%%%%%%%%%%%%%%%%%%%%%%%%%%%%



{\color{white}x}

\label{DocLastPage}

\end{document}
